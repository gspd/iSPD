% exemplo do GSPD com LaTeX!

\documentclass[12pt]{article}

\usepackage{graphicx} %para adicionar imagem ao texto

\usepackage[brazil]{babel}
\usepackage[latin1]{inputenc}
\usepackage{ae}

\title{Gram�tica do gerador de escalonamento}
\author{Denison Menezes}

\begin{document}

\maketitle % escreve o titulo

<d�gito>	::= 0 | 1 | 2 | 3 | 4 | 5 | 6 | 7 | 8 | 9

<letra>		::=	a | b | c | ... | z | A | B | C | ... | Z

<vari�vel>	::= <vari�vel\_tarefa> | <vari�vel\_recurso>

<vari�vel\_tarefa>:= [TCP] | [NTS] | [NTC] | [NTA] | [TC] | [PCU]

<vari�vel\_recurso>:= [PP] | [LC] | [TCT] | [NTE] | [TCMT]

<inteiro>	::=	\{<d�gito>\}$^{+}$

<real>		::=	\{<inteiro>\}$^{+}$<ponto>\{<inteiro>\}$^{+}$

<identificador>	::=	<letra> \{<letra> | <d�gito>\}$^{*}$

<constante>	::=	<inteiro> | <real>

<fila>		::= 

<formula>	::=	``RANDOM'' | (``CRESCENT'' | ``DECREASING'') ``('' <express�o> ``)'' | <fila>

<express�o>	::=	<numExp>

<numExp>	::=	<numExp2> ( ( ``-'' | ``+'' ) <numExp2> )$^{*}$

<numExp2>	::=	<numExp3> ( ( ``/'' | ``*'' ) <numExp3> )$^{*}$

<numExp3>	::=	[ ( ``+'' | ``-'' ) ] numExp4

<numExp4>	::=	<vari�vel> | <constante> | ``('' <express�o> ``)''

<inicial>	::=	<nome> <caracteristica> <escalonador\_tarefa> <escalonador\_recurso>

<nome>		::= ``SCHEDULER'' <identificador>

<caracteristica>	::=	``STATIC'' | ``DYNAMIC'' <tipo\_atualizacao>

<escalonador\_tarefa>	::=	``TASK SCHEDULER:'' <formula>

<escalonador\_recurso>	::=	``RESOURCE SCHEDULER:'' <formula>

<tipo\_atualizacao>	::=	``TASK ENTRY'' | ``TASK OUTPUT'' | ``TIME INTERVAL ``('' <real> ``)''

\end{document}